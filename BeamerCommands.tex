%!TEX root = tutorial01-splitshow.tex

\usepackage{wasysym} % for \smiley \frownie
%
% \usepackage{wrapfig}
% \begin{wrapfigure}{POS}{WIDTH}{LINES TO RESERVE}
% ..
% \end{wrapfigure}
% The last argument is optional

% Figures need exact positioning in columns
% Use the following command to place them
\newcommand{\figincol}[1]{
\pgfputat{\pgfxy(0,0)}{\pgfbox[left,top]{#1}}
}

% ALSO:
% \pgfputat {\pgfxy(XX,YY)}{\pgfbox[left,base]{#1}}
% 
% LL = (0cm,-7cm) 
% UR = (11cm,1cm)
%
% pgfdeclareimage
% pgfuseimage

\newcommand{\Ra}{\Rightarrow}
%% BEAMER SPECIFIC COMMANDS

\newcommand{\bi}{\begin{itemize}}
\newcommand{\ei}{\end{itemize}}
\newcommand{\bc}{\begin{center}}
\newcommand{\ec}{\end{center}}


\setbeamercolor{math text}{fg=blue!50!normal text.fg}
\newcommand{\animframe}[2]{\begin{frame}[<+->]{#1}#2\end{frame}}
\newcommand{\animframesq}[2]{\begin{frame}[<+->][shrink,squeeze]{#1}#2\end{frame}}
\newcommand{\animframejsq}[2]{\begin{frame}[<+->][squeeze]{#1}#2\end{frame}}

\newcommand{\animframen}[2]{\begin{frame}[<+->]{#1}#2\emptynote\end{frame}}
\newcommand{\animframesqn}[2]{\begin{frame}[<+->][shrink,squeeze]{#1}#2\emptynote\end{frame}}
\newcommand{\animframejsqn}[2]{\begin{frame}[<+->][squeeze]{#1}#2\emptynote\end{frame}}

\newcommand{\myframe}[2]{\begin{frame}{#1}#2\end{frame}}
\newcommand{\myframesq}[2]{\begin{frame}[shrink,squeeze]{#1}#2\end{frame}}
\newcommand{\myframejsq}[2]{\begin{frame}[squeeze]{#1}#2\end{frame}}

\newcommand{\myframen}[2]{\begin{frame}{#1}#2\emptynote\end{frame}}
\newcommand{\myframesqn}[2]{\begin{frame}[shrink,squeeze]{#1}#2\emptynote\end{frame}}
\newcommand{\myframejsqn}[2]{\begin{frame}[squeeze]{#1}#2\emptynote\end{frame}}


%\setbeamertemplate{footline}[frame number]
\newtheorem{Solution}[theorem]{Solution}
\newtheorem{Comm}[theorem]{Comment}
\newtheorem{Note}[theorem]{Note}

\newcommand{\bcol}[1][t]{\begin{columns}[#1]} % optional argument: alignment (t,b,c)
\newcommand{\ecol}{\end{columns}}
\newcommand{\col}[1][0.5\textwidth]{\column{#1}} % argument: width of the column
%\parindent = 10pt
\newcommand{\scaletext}[3]{ % scale-factor original-width TEXT
\scalebox{#1}{\begin{minipage}[h]{#2\textwidth} #3 \end{minipage}
}}

% note: beamer slides are 128mm by 96 mm
\newcommand{\putatUL}[4]{ % width xpos ypos WHAT; upper left corner is put at the said pos
\begin{textblock*}{#1}[0,0](#2,#3)
#4
 \end{textblock*}
}
\newcommand{\putatBR}[4]{ % width xpos ypos WHAT; bottom right corner is put at the said pos
\begin{textblock*}{#1}[1,1](#2,#3)
#4
 \end{textblock*}
}
\newcommand{\putatBL}[4]{ % width xpos ypos WHAT; bottom left corner is put at the said pos
\begin{textblock*}{#1}[0,1](#2,#3)
#4
 \end{textblock*}
}
\newcommand{\putatUR}[4]{ % width xpos ypos WHAT; bottom right corner is put at the said pos
\begin{textblock*}{#1}[1,0](#2,#3)
#4
 \end{textblock*}
 }
 \newcommand{\putatMID}[4]{ % width xpos ypos WHAT; bottom right corner is put at the said pos
\begin{textblock*}{#1}[0.5,0.5](#2,#3)
#4
 \end{textblock*}
 }

\newcommand{\putat}[3]{\begin{picture}(0,0)(0,0)\put(#1,#2){#3}\end{picture}} % xrelpos yrelpos WHAT

\makeatletter
\newcommand{\insertprevframe}[1]{
	\def\beamer@origlmargin{\Gm@lmargin}
%    \vbox{\hfill\insertslideintonotes{0.125}\hskip-\Gm@rmargin\hskip0pt%
%      \vskip-0.125\paperheight\nointerlineskip}%
	\insertslideintonotes{#1}
}

%\newcommand{\insertslideintonotes}[1]{{%
%  \begin{pgfpicture}{0cm}{0cm}{#1\paperwidth}{#1\paperheight}
%    \begin{pgflowlevelscope}{\pgftransformscale{#1}}%
%      \color[gray]{0.8}
%      \pgfpathrectangle{\pgfpointorigin}{\pgfpoint{\paperwidth}{\paperheight}}
%      \pgfusepath{fill}
%      \color{black}
%      {\pgftransformshift{\pgfpoint{\beamer@origlmargin}{\footheight}}\pgftext[left,bottom]{\copy\beamer@frameboxcopy}}
%    \end{pgflowlevelscope}
%  \end{pgfpicture}%
%  }}
\makeatother
% For adding items to the notes pages (we do not want animation there)
\newcommand{\bin}{\bi[<1->]}
