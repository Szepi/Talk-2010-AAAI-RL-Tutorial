%!TEX root = tutorial01-splitshow.tex


\usepackage[textwidth=\marginparwidth]{todonotes}
\usepackage{pdfsync}
\usepackage{hyperref}
\usepackage{fancybox}

% For citations
\newif\ifnumcites
%\numcitestrue
\numcitesfalse




% The paper has an extended (long) version and a short version
\newif\iflong % Sometimes we want to keep two versions; a short and a long one -- this is useful for that..
%\longfalse
\longtrue

% Turn on/off notes and descriptions of research problems
\newif\ifcomm
%\commfalse % also turns off internal todo comments
%\commtrue

% Turn on/off internal todo comments
\newif\iftodo
\todofalse
%\todotrue


\ifnumcites
  \usepackage[numbers]{natbib}
  \bibliographystyle{plainnat}
\else
  \usepackage{natbib}
  \bibliographystyle{apalike}
\fi
%\bibliographystyle{apalike}
% plain, acm, ieeetr, alpha, acm, abbrv, siam
% plainnat.bst, abbrvnat.bst and unsrtnat.bst
% http://web.reed.edu/cis/help/LaTeX/bibtexstyles.html

\if0
\newcommand{\citep}[1]{\cite{#1}}
\newcommand{\citet}[1]{\cite{#1}}
\newcommand{\citealt}[1]{\cite{#1}}
\newcommand{\iftextcite}[1]{}

\newcommand{\npcite}[1]{\cite{#1}}
\newcommand{\yrcite}[1]{\cite{#1}}
\fi



%\usepackage{amsthm}

\usepackage{amssymb}
%\usepackage[dvips]{graphics}
\usepackage{amsmath,amsthm,amsfonts} % Learn about the AMS package, again very useful!

\usepackage{graphicx}
\usepackage{epstopdf}
\usepackage{stmaryrd}
\usepackage{dsfont}
%\usepackage{small-headings}


\newif\ifshort
\iflong
	\shortfalse
\else
	\shorttrue
\fi


% THEOREMS -------------------------------------------------------
\theoremstyle{plain}
\newtheorem{thm}{Theorem}
\newtheorem{cor}[thm]{Corollary}
\newtheorem{lem}[thm]{Lemma}
\newtheorem{prop}[thm]{Proposition}
\newtheorem{conj}[thm]{Conjecture}
\newtheorem{proofthm}{Proof of Theorem 2}

%\renewtheorem{definition}[thm]{Definition}
\theoremstyle{definition}
\newtheorem{defn}{Definition}
%\theoremstyle{remark}


\newtheoremstyle{example}% ?name? 
{3pt}%	?Space above? 
{3pt}%	?Space below? 
{\itshape}%	?Body font?
{}%	?Indent amount?1 
{}% ?Theorem head font? 
{:}%	?Punctuation after theorem head? 
{.5em}%	?Space after theorem head?2 
{}%
\theoremstyle{example}
\newtheorem{ex}{Example}
%\newtheorem{fact}{Fact}
\newtheorem{rem}{Remark}

\newcounter{assumption}%[section]
\newcommand{\theassumptionletter}{A}
\renewcommand{\theassumption}{\theassumptionletter\arabic{assumption}}

\newenvironment{ass}[1][]{\begin{trivlist}\item[] \refstepcounter{assumption}%
 {\bf Assumption\ \theassumption\ #1} }{%\par\nobreak\noindent\sl\ignorespaces}{%
 \ifvmode\smallskip\fi\end{trivlist}}
\newcommand{\aref}[1]{(\ref{#1})}
\newenvironment{ass*}[1][]{\begin{trivlist}\item[] %
 {\bf Assumption\  #1} }{%\par\nobreak\noindent\sl\ignorespaces}{%
 \ifvmode\smallskip\fi\end{trivlist}}


%\newenvironment{remark}
\newtheorem{remark}{Remark}

%\newenvironment{proof}{{\bf Proof.}}{\hfill\rule{2mm}{2mm}\\}

% Keep whatever you need from here


\newcommand{\norm}[1]{\left\Vert#1\right\Vert}
\newcommand{\smallnorm}[1]{\|#1\|}
\newcommand{\abs}[1]{\left\vert#1\right\vert}
\newcommand{\supnorm}[1]{\norm{#1}_\infty}

\newcommand{\set}[1]{\left\{#1\right\}}
\newcommand{\cset}[2]{\left\{\,#1\,:\,#2\,\right\}}

\renewcommand{\natural}{\mathbb N}                   % Natural numbers
\newcommand{\Real}{\mathbb R}                        % Real numbers
\newcommand{\real}{\mathbb R}                        % again..
\newcommand{\R}{{\mathbb{R}}}                        % again..

\newcommand{\Prob}[1]{{\mathbb P}\left(#1\right)}    % Probabilities; example: \Prob{X>\eps}<1-\delta
\renewcommand{\P}{{\mathbb P}}                         % Probabilities when we want to control the parenthesis
\newcommand{\EE}[1]{{\mathbb E}\left[#1\right]}      % Expectations
\newcommand{\E}{{\mathbb E}}                         % Expectations  when we want to control the parenthesis
\newcommand{\Var}[1]{{\mathrm{Var}}\left[#1\right]}  % Variances
%\newcommand{\one}{\mathbb I}
\newcommand{\one}[1]{{\mathbb I}_{\{#1\}}}           % Characteristic function

\newcommand{\MB}{\mathcal{B}}
\newcommand{\MA}{\mathcal{A}}
\newcommand{\MS}{\mathcal{S}}
\newcommand{\MF}{\mathcal{F}}
\newcommand{\MC}{\mathcal{C}}
\newcommand{\MRR}{\mathcal{R}}
\newcommand{\MD}{\mathcal{D}}
\newcommand{\MP}{\mathcal{P}}
\newcommand{\MU}{\mathcal{U}}
\newcommand{\MO}{\mathcal{O}}
\newcommand{\MX}{\mathcal{X}}
\newcommand{\GG}{\mathcal{G}}
\newcommand{\hZ}{\hat{Z}}
\newcommand{\hF}{\hat{F}}
\newcommand{\hL}{\hat{L}}
\newcommand{\tL}{\tilde{L}}

\newcommand{\MI}{{\bf I}} %\mathbb{I}}


\newcommand{\eps}{\varepsilon}                       % Nice epsilon
\newcommand{\ep}{\varepsilon}                        % Shorthand for nice epsilon
\newcommand{\de}{\delta}                             % Shorthand for delta
\newcommand{\To}{\longrightarrow}
\newcommand{\ra}{\rightarrow}

\newcommand{\argmin}{\mathop{\rm argmin}}
\newcommand{\argmax}{\mathop{\rm argmax}}
\newcommand{\diag}{\mathop{\rm diag}}
\newcommand{\inlinemin}{\wedge}
\newcommand{\inlinemax}{\vee}

\newcommand{\ip}[2]{\langle #1,#2\rangle}
\newcommand{\bigip}[2]{\Big\langle #1,#2\Big\rangle}
\newcommand{\eqdef}{\stackrel{\mbox{\rm\tiny def}}{=}}
\newcommand{\aP}{{\cal P}}

% Shorthands I use for math environments
\newcommand{\beq}{\begin{equation}}
\newcommand{\eeq}{\end{equation}}
\newcommand{\beqa}{\begin{eqnarray}}
\newcommand{\eeqa}{\end{eqnarray}}
\newcommand{\beqan}{\begin{eqnarray*}}
\newcommand{\eeqan}{\end{eqnarray*}}
\newcommand{\ben}{\begin{eqnarray*}}
\newcommand{\een}{\end{eqnarray*}}

\newcommand{\RA}{$\Rightarrow$}

\newcommand{\TODO}[2][]{\todo[#1]{#2}}

\iflong
\else
	\renewcommand{\note}[2][]{}
	\renewcommand{\bibnote}[2][]{}
	\renewcommand{\problem}[2][]{}
\fi

\ifcomm
   \newcommand\comm[1]{\textcolor{blue}{ #1}}
\else
   \newcommand\comm[1]{}
%   \renewcommand{\todo}[1]{}
   \renewcommand{\todo}[2][]{}
\fi

\iftodo
\else
  \renewcommand{\TODO}[2][]{}
\fi

\newcommand{\remove}[1]{\textcolor{blue}{\sout{#1}}}
\newcommand{\tO}{\tilde{O}}

% Boldfaced lowercase greek letters as described at
% http://www.rpi.edu/dept/acs/rpinfo/common/Computing/Consulting/Software/LaTeX/Hints/Greek_Chars.html
\def\bmath#1{\mbox{\boldmath$#1$}}
% http://www.ctan.org/tex-archive/info/symbols/comprehensive/symbols-a4.pdf page 68
\newcommand\independent{\protect\mathpalette{\protect\independenT}{\perp}}
\def\independenT#1#2{\mathrel{\rlap{$#1#2$}\mkern2mu{#1#2}}}


\newcommand{\Set}{S}
\newcommand{\States}{\mathcal{X}}
\newcommand{\Actions}{\mathcal{A}}
\newcommand{\TranKernel}{\mathcal{P}}
\newcommand{\JTranKernel}{\mathcal{P}_{0}}
\newcommand{\PKernel}{\mathcal{P}}
\newcommand{\RKernel}{\mathcal{Q}}
\newcommand{\st}{x}
\newcommand{\St}{X}
\renewcommand{\action}{a}
\newcommand{\nextaction}{a'}
\newcommand{\Action}{A}
\newcommand{\Nextaction}{A'}
\newcommand{\reward}{r}
\newcommand{\Reward}{R}
\newcommand{\Ret}{\mathcal{R}}
\newcommand{\MDP}{\mathcal{M}}
\newcommand{\nextstate}{y}
\newcommand{\Nextstate}{Y}
\newcommand{\rewardfun}{r}
\newcommand{\TransPOp}{P}
\newcommand{\id}{I}

\renewcommand{\atop}{^{\top}}
\newcommand{\SA}{\States\times\Actions}


\newcommand{\astate}{z}
\newcommand{\AStates}{Z}%\State_A}
\newcommand{\APKernel}{\PKernel_A}
\newcommand{\rewardrange}{{\mathcal R}}
\iflong
\newcommand{\myfootnote}[1]{\footnote{#1}}
\else
\newcommand{\myfootnote}[1]{}
\fi
\renewcommand{\epsilon}{\varepsilon}

\newcommand{\tder}{\nabla_\theta}
\newcommand{\tders}{\nabla_\theta}
\newcommand{\pder}{\frac{\partial}{\partial \theta}}
\newcommand{\pders}{\tfrac{\partial}{\partial \theta}}


\newcommand{\pai}{{(i)}}
\newcommand{\integer}{\mathbb{Z}}
\newcommand{\cD}{{\cal D}}
\newcommand{\cZ}{{\cal Z}}
\newcommand{\cX}{{\cal X}}
\newcommand{\cW}{{\cal W}}

\newcommand{\cG}{{\cal G}}
\newcommand{\cF}{{\cal F}}
\newcommand{\cH}{{\cal H}}
\renewcommand{\phi}{\varphi}
\newcommand{\ewithin}{\eps_{\text{W}}}
\newcommand{\ebetween}{\eps_{\text{B}}}
\newcommand{\emean}{\eps_{\text{M}}}
\DeclareMathOperator{\trace}{trace}
\newcommand{\e}{\mathbf{1}}
\renewcommand{\eps}{\varepsilon}
\newcommand{\cM}{{\cal M}}
\newcommand{\cS}{{\cal S}}
\newcommand{\tM}{\tilde{M}}
\newcommand{\MRP}{{\cal M}}
\newcommand{\kfun}{\mathbb{K}}
\newcommand{\cK}{{\cal K}}

\newcommand{\aparam}{\omega}
\newcommand{\dimaction}{d_{\Actions}}
\newcommand{\dimaparam}{d_{\aparam}}
\newcommand{\traj}{\xi}
\newcommand{\Trajset}{\Xi}
\newcommand{\Traj}{X}
\newcommand{\perf}{\rho}
\newcommand{\dstat}{\mu} % stationary distribution underlying a Markov chain
\newcommand{\Regret}{{\bf R}}
\renewcommand{\AA}{{\cal A}}
%\newcommand{\SA}{\States\times\Actions}
\newcommand{\Sample}{{\cal D}}
\newcommand{\hA}{\hat{A}}
\newcommand{\hb}{\hat{b}}
\newcommand{\ZZ}{{\cal Z}}
\newcommand{\ttop}{^\top}
\newcommand{\FF}{{\cal F}}
\newcommand{\PiStat}{\Pi_{\rm stat}}
\newcommand{\td}{\delta}
\newcommand{\hV}{\hat{V}}
\newcommand{\mynote}[1]{}
\newcommand{\elg}{z}
\newcommand{\furtherreading}{}
\newcommand{\rfun}{\reward}
\newcommand{\pscorefun}[1]{\frac{\partial \ln \pi_\aparam(#1)}{\partial \aparam}}
\newcommand{\pscorefunp}[1]{\frac{\partial}{\partial\aparam}\,\log \pi_\aparam(#1)}
\newcommand{\scorefun}{\psi}
\newcommand{\hQ}{\hat{Q}}
\renewcommand{\th}{^{\rm th}}
\newcommand{\bee}{\begin{enumerate}}
\newcommand{\eee}{\end{enumerate}}

\usepackage{algorithm}
\usepackage{algpseudocode}
\algnewcommand\algorithmicto{\textbf{to}}
\algnewcommand\algorithmicdownto{\textbf{downto}}
